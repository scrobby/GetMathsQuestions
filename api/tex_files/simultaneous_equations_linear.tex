\documentclass{article}
\usepackage{pythontex}
\pythontexcustomc[begin]{py}{from numpy import *; import random}

\begin{pycode}
# INSTRUCTIONS ARE AFTER # SYMBOL
coeffs_range = [-8, 8] # <- this is the range of coefficients in questions

generate_random_solutions = 1 # <- make this 0 to manually input answers

answers_range = [-12, 12] # <- range of answers if automatically generated

number_of_questions = 100 # <- number of questions if automatically generated

if generate_random_solutions:
  x = [random.randint(answers_range[0], answers_range[1]) for i in range(number_of_questions)]
  y = [random.randint(answers_range[0], answers_range[1]) for i in range(number_of_questions)]
else:
  x = [2, -1, 1, 6, -5, 2, -3, 10, -7, -4, -6] # if manual answers
  y = [-3, 5, 5, -2, 9, 4, -6, 8, -20, 4, 1] # if manual answers

\end{pycode}

\begin{document}



Solve for $x$, $y$:
\newline

\begin{pycode}

for ieqn in range(len(x)):
  coeffs = [random.randint(coeffs_range[0], coeffs_range[1]) for i in range(4)]
  signs = [None] * len(coeffs)
  for c in range(len(coeffs)):
    if coeffs[c] < 0:
      signs[c] = " - "
    else:
      signs[c] = " + "
  ans1 = coeffs[0]*x[ieqn] + coeffs[1]*y[ieqn]
  ans2 = coeffs[2]*x[ieqn] + coeffs[3]*y[ieqn]

  print("\n" + str(ieqn+1) + ") ")

  print(str(coeffs[0]) + r'\textit{x}' + signs[1] + str(abs(coeffs[1])) + r'\textit{y}' + " = " + str(ans1) + ",   ")

  print(str(coeffs[2]) + r'\textit{x}' + signs[3] + str(abs(coeffs[3])) + r'\textit{y}' + " = " + str(ans2))

  for i in range(7):
    print(r'\textit{\newline}')

# divide answers by coefficients 

\end{pycode}

\pagebreak

Solutions:
\newline

\begin{pycode}
for ieqn in range(len(x)):
  print("\n" + str(ieqn+1) + ") ")
  print(r'\textit{x}' + " = " + str(x[ieqn]) + ", ")
  print(r'\textit{y}' + " = " + str(y[ieqn]))
\end{pycode}


\end{document}
