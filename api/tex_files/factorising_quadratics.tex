\documentclass{article}
\usepackage{pythontex}
\usepackage{fancyhdr}
\usepackage{hyperref}

\renewcommand{\footrulewidth}{0.5pt}

\pythontexcustomc[begin]{py}{from numpy import *; import random; import json}

\pagestyle{fancy}
\rfoot{\thepage}
\lfoot{Questions generated by \href{http://getmathsquetsions.com/}{GetMathsQuestions.com}}

\begin{pycode}

equals_range = [0,0]
answers_range = [0, 0]
coeffs_range = [0, 0]

with open('factorising_quadratics_data.json') as json_file:
  data = json.load(json_file)
  number_of_questions = data['number_of_questions']
  answers_range[0] = data['answers_range_low']
  answers_range[1] = data['answers_range_high']
  coeffs_range[0] = data['coeffs_range_low']
  coeffs_range[1] = data['coeffs_range_high']
  equals_range[0] = data['equals_range_low']
  equals_range[1] = data['equals_range_high']


a = [random.choice(coeffs_range) for i in range(number_of_questions)]

x1 = [random.randint(answers_range[0], answers_range[1]) for i in range(number_of_questions)]

x2 = [random.randint(answers_range[0], answers_range[1]) for i in range(number_of_questions)]

equals = [random.randint(equals_range[0], equals_range[1]) for i in range(number_of_questions)]

\end{pycode}

\begin{document}

\noindent\textbf{Solve for $x$:}
\newline
\newline

\begin{pycode}

for ieqn in range(len(a)):

  b = a[ieqn] * (-x1[ieqn] - x2[ieqn])
  c = (a[ieqn] * -(x1[ieqn]) * -(x2[ieqn])) + equals[ieqn]
  
  if b <0:
    bsign = " - "
  else:
    bsign = " + "
  if c < 0:
    csign = " - "
  else:
    csign = " + "

  print("\n" + str(ieqn+1) + ") ")
  if a[ieqn] == 1:
    print(r'\textit{$x^2$}' + bsign + str(abs(b)) + r'\textit{$x$}' + csign + str(abs(c)) + " = " + str(equals[ieqn]))
  else:
    print(str(a[ieqn]) + r'\textit{$x^2$}' + bsign + str(abs(b)) + r'\textit{$x$}' + csign + str(abs(c)) + " = " + str(equals[ieqn]))

  for i in range(7):
    print(r'\textit{\newline}')

\end{pycode}

\pagebreak

Solutions:
\newline

\begin{pycode}
for ieqn in range(len(a)):
  print("\n" + str(ieqn+1) + ") ")
  print(r'\textit{x}' + " = " + str(x1[ieqn]) + ", ")
  print(r'\textit{x}' + " = " + str(x2[ieqn]))
\end{pycode}

\end{document}