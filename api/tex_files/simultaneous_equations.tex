\documentclass{article}
\usepackage{pythontex}
\usepackage{fancyhdr}
\usepackage{hyperref}

\renewcommand{\footrulewidth}{0.5pt}

\pythontexcustomc[begin]{py}{from numpy import *; import random; import json}

\pagestyle{fancy}
\rfoot{\thepage}
\lfoot{Questions generated by \href{http://getmathsquetsions.com/}{GetMathsQuestions.com}}



\begin{pycode}

answers_range = [0, 0]
coeffs_range = [0, 0]

with open('simultaneous_equations_data.json') as json_file:
    data = json.load(json_file)
    number_of_questions = data['number_of_questions']
    answers_range[0] = data['answers_range_low']
    answers_range[1] = data['answers_range_high']
    coeffs_range[0] = data['coeffs_range_low']
    coeffs_range[1] = data['coeffs_range_high']

x = [random.randint(answers_range[0], answers_range[1]) for i in range(number_of_questions)]
y = [random.randint(answers_range[0], answers_range[1]) for i in range(number_of_questions)]

\end{pycode}

\begin{document}

\noindent\textbf{Solve for $x$, $y$:}
\newline
\newline
\newline

\begin{pycode}

for ieqn in range(len(x)):
  coeffs = [random.randint(coeffs_range[0], coeffs_range[1]) for i in range(4)]
  signs = [None] * len(coeffs)
  for c in range(len(coeffs)):
    if coeffs[c] < 0:
      signs[c] = " - "
    else:
      signs[c] = " + "
  ans1 = coeffs[0]*x[ieqn] + coeffs[1]*y[ieqn]
  ans2 = coeffs[2]*x[ieqn] + coeffs[3]*y[ieqn]


  print(r'\noindent\begin{minipage}{\textwidth}')
  print("\n" + str(ieqn+1) + ") ")
  print(r'\textit{\newline}')
  print(str(coeffs[0]) + r'\textit{x}' + signs[1] + str(abs(coeffs[1])) + r'\textit{y}' + " = " + str(ans1))
  print(r'\textit{\newline}')
  print(str(coeffs[2]) + r'\textit{x}' + signs[3] + str(abs(coeffs[3])) + r'\textit{y}' + " = " + str(ans2))

  for i in range(7):
    print(r'\textit{\newline}')
  print(r'\end{minipage}')

\end{pycode}

\pagebreak

Solutions:
\newline

\begin{pycode}
for ieqn in range(len(x)):
  print("\n" + str(ieqn+1) + ") ")
  print(r'\textit{x}' + " = " + str(x[ieqn]) + ", ")
  print(r'\textit{y}' + " = " + str(y[ieqn]))
\end{pycode}


\end{document}
